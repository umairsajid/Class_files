\documentclass[11pt]{article}
\usepackage[margin=1in,footskip=0.25in]{geometry}
\usepackage{amsmath}
\usepackage{ amssymb }
%Gummi|065|=)
\title{\textbf{Assignment 0}}
\author{Sterling Kohel}
\date{\today}
\begin{document}

\maketitle


\section{What is the  32-bit complement  representation of the number -10117?}

\hspace{2ex} $10117_{10}$  = $0000 0000 0000 0000 0010 0111 1000 0101_{2}$

flip bit  = $1111 1111 1111 1111 1101 1000 0111 1010_{2}$

\hspace{1.3ex}add 1 = $1111 1111 1111 1111 1101 1000 0111 1011_{2}$

\section{Prove the following statement:  Assume x and y are positive integers.  If the digits of y are the digits of x, but just rearranged, then (x - y) mod 9 = 0.}

\hspace{3.2ex}Let C = some constant.

x - C = multiple of 9

Since y contains the same digits as x, y - C = multiple of 9.

(x - C - (y - C)) mod 9 = 0

(x - C - y + C) mod 9 = 0

(x - y) mod 9 = 0 $\square$

\section{Is the converse of the previous statement, which you just proved, true?  If so, prove it.  If not, provide a concrete counterexample.  In either case you must clearly state what the converse actually is.}

The converse of 2 is:

 if (x - y) mod 9 = 0 then the digits of y are the digits of x, but just rearranged 
 
 Counterexample:
 
 x = 45
 
 y = 9
 
 (45 - 9) mod 9 = 0 
 however 9 is not a rearrangment of 45.

\end{document}
